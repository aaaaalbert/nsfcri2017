%!TEX root = proposal.tex

\section*{Summary}
\explain{1 page max.}
\explain{The Project Summary should be written in the third person, informative to other persons working in the same or related fields, and, insofar as possible, understandable to a scientifically or technically literate lay reader. It should not be an abstract of the proposal.}


\subsection*{Overview}
\explain{The overview includes a description of the activity that would result if the proposal were funded and a statement of objectives and methods to be employed.}
\albert{What we need to do to enable the project goals and broader impacts.}


\subsection*{Intellectual Merit}
\explain{The statement on intellectual merit should describe the potential of the proposed activity to advance knowledge.}
\albert{Two possible things. One, merit in the actual work --- this is understanding the current shortcomings and designing transformations, particularly a methodological approach. Two, merit in the outcomes --- much like Broader Impacts I suppose.}


\subsection*{Broader Impacts}
\explain{The statement on broader impacts should describe the potential of the proposed activity to benefit society and contribute to the achievement of specific, desired societal outcomes.}
\albert{Utopia. Describe the bright prospective future enabled by our system. Ideas: Sensor owners control the data, and get to decide which data to offer to whom at what accuracy and conditions. Brokers may be used to mediate supply and demand. Systems are interoperable. Vendors compete on features. The old vertically-integrated per-vendor full-stack silo model has been obsoleted. Collaborative use cases are developed, and open new data sources.}


\begin{description}
  \item{Keywords:} \todo{3-5 high-level keyword descriptors; separated by semi-colons}
\end{description}

\newpage

