%!TEX root = proposal.tex

\section*{Summary}
\explain{1 page max.}
\explain{The Project Summary should be written in the third person, informative to other persons working in the same or related fields, and, insofar as possible, understandable to a scientifically or technically literate lay reader. It should not be an abstract of the proposal.}


\subsection*{Overview}
\explain{The overview includes a description of the activity that would result if the proposal were funded and a statement of objectives and methods to be employed.}
\albert{What we need to do to enable the project goals and broader impacts.}


The \acrfull{IoT} paradigm proposes networked embedded devices
for large-scale, distributed sensing, control, and computational tasks,
with the Internet serving as part of the networking substrate for
the devices (``Things'') to exchange information.
In its current realization today, the \acrlong{IoT} comprises
devices in cars, homes and other buildings, public transport,
industrial installations, medical contexts, and farming.
This enables remote control and supervision of systems, provides
means of data collection for evaluation, aids
task automation, and advances convergence of cyber-physical systems.

However, the larger-scale usefulness of current \acrlong{IoT}
systems is held back by design choices of system vendors to
integrate all of the functionality (from sensing and recording
to data exchange and analysis) into monolithic silos of services.
This design not only grants the vendor exclusive access to the
sensor data, but also allows them to exert control over the ways
that devices are used and operated. For example, a firmware update
may disable device functionality (\albert{ref}), or services could
require paid-for subscriptions for the device to work
(\albert{ref}).
If the device owner does not comply, their ability of using
the device might be hampered. Furthermore, the owner has little
choice over which type of device they interlink with what service
provider.

Besides causing inefficiency and loss of control for the end user,
these issues also present substantial hurdles for the progress
of scientific research on distributed sensor devices.
The inability of accessing sensor data precludes using the
data for novel applications; fragmentation and per-vendor silos
prohibit studying cooperative use cases.

Therefore, this proposal suggests to re-think the design choices
that made the current acrlong{IoT} landscape a heap of segregated
monoliths, and instead make it an an open system with proper interfaces and people in control to enable cooperative sensing, crowdery, etc.

\subsection*{Intellectual Merit}
\explain{The statement on intellectual merit should describe the potential of the proposed activity to advance knowledge.}
\albert{Two possible things. One, merit in the actual work --- this is understanding the current shortcomings and designing transformations, particularly a methodological approach. Two, merit in the outcomes --- much like Broader Impacts I suppose.}


\subsection*{Broader Impacts}
\explain{The statement on broader impacts should describe the potential of the proposed activity to benefit society and contribute to the achievement of specific, desired societal outcomes.}
\albert{Utopia. Describe the bright prospective future enabled by our system. Ideas: Sensor owners control the data, and get to decide which data to offer to whom at what accuracy and conditions. Brokers may be used to mediate supply and demand. Systems are interoperable. Vendors compete on features. The old vertically-integrated per-vendor full-stack silo model has been obsoleted. Collaborative use cases are developed, and open new data sources.}


\begin{description}
  \item{Keywords:} \todo{3-5 high-level keyword descriptors; separated by semi-colons}
\end{description}

\newpage

