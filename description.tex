%!TEX root = proposal.tex

\section{Description}

\explain{15 pages max.}

\explain{Expectations: Document the need for the new infrastructure in light of existing infrastructure available to the relevant CISE research communities. Demonstrate CISE research community support within the participating institutions. Show how the infrastructure will enable innovative CISE disciplinary research.}

\explain{Characteristics: Infrastructure benefits a broad-based community of CISE researchers that extends well beyond awardee institutions. Proposal includes outreach to communities and commitment to high-quality service. Infrastructure enables compelling research for CISE researchers that may extend the frontiers of the core CISE disciplines.}

\explain{The Project Description should provide a clear statement of the work to be undertaken and must include the objectives for the period of the proposed work and expected significance; the relationship of this work to the present state of knowledge in the field, as well as to work in progress by the PI under other support.\\
The Project Description should outline the general plan of work, including the broad design of activities to be undertaken, and, where appropriate, provide a clear description of experimental methods and procedures. Proposers should address what they want to do, why they want to do it, how they plan to do it, how they will know if they succeed, and what benefits could accrue if the project is successful. The project activities may be based on previously established and/or innovative methods and approaches, but in either case must be well justified. These issues apply to both the technical aspects of the proposal and the way in which the project may make broader contributions.}


\explain{
\begin{itemize}
  \item Proposed CISE research infrastructure and its estimated lifetime, noting whether it is new infrastructure to be created and operated or existing infrastructure to be enhanced and operated;
  \item Compelling new CISE research opportunities enabled by the proposed infrastructure (including a description of the steps taken to identify the research opportunities enabled by the infrastructure as well as evidence that a diverse community of users plan to use the capabilities provided);
  \item CISE sub-disciplines that will benefit from the infrastructure and CISE-centric research groups within the participating institutions that will use the infrastructure;
  \item Existing related resources along with a justification that the proposed research cannot be accomplished with these resources at the institution or elsewhere;
  \item Samples of focused research projects or agendas that the infrastructure will enable (note that the novelty and innovative aspects of the research must be evident along with clear evidence that the proposed infrastructure is essential to moving CISE research frontiers forward);
  \item Quality of service commitment to the relevant CISE research community;
  \item Means by which user satisfaction will be evaluated and used to refine and improve subsequent infrastructure operations;
  \item Plans for outreach to ensure that a broad community of users is engaged;
  \item Community plans to provide long-term sustainability of the infrastructure;
  \item Qualifications of the PIs and the project team to manage the creation or enhancement and operations of the research infrastructure in support of its users; and
  \item Detailed project management plan, including a timeline, that outlines all steps to be undertaken to acquire, develop, and/or operate the research infrastructure, and identify the parties responsible for each major task.
\end{itemize}
}


\subsection{Broader Impact}
\explain{This section should provide a discussion of the broader impacts of the proposed activities. Broader impacts may be accomplished through the research itself, through the activities that are directly related to specific research projects, or through activities that are supported by, but are complementary to the project. NSF values the advancement of scientific knowledge and activities that contribute to the achievement of societally relevant outcomes. Such outcomes include, but are not limited to: full participation of women, persons with disabilities, and underrepresented minorities in science, technology, engineering, and mathematics (STEM); improved STEM education and educator development at any level; increased public scientific literacy and public engagement with science and technology; improved well-being of individuals in society; development of a diverse, globally competitive STEM workforce; increased partnerships between academia, industry, and others; improved national security; increased economic competitiveness of the US; and enhanced infrastructure for research and education.}


\subsection{Results from Prior NSF Support}
\explain{If any PI or co-PI identified on the project has received NSF funding (including any current funding) in the past five years, the Project Description must contain information on the award(s), irrespective of whether the support was directly related to the proposal or not.}


\newpage

