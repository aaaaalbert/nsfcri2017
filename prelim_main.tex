%!TEX root = preliminary_proposal.tex

\section{Project Description}

(Two pages overall)

\begin{description}
  \item[Project Title:]
  \item[Project Type:] II-New, II-EN, CI-P, CI-New, CI-EN, or CI-SUSTAIN
  \item[Collaborative Partners:] All collaborative partners should be identified, including PIs and Co-PIs of the collaborative partners with their institutional and departmental affiliations
  \item[CISE core division:] (CCF, CNS, IIS) most relevant for the submission – the CISE division the proposal will be submitted to
  \item[Keywords:] Three keyword descriptors for the research focus that the requested infrastructure will enable.
\end{description}


\subsection{Overview}
A concise description of the infrastructure to be developed, enhanced, or
sustained. This includes a description of major equipment needs for the 
project as well as other significant costs. Projects that involve 
enhancements to, or sustainment of, existing infrastructures should 
include information about the existing resources.

\subsection{CISE Research Focus}
This section describes the research focus that is enabled by the 
infrastructure, the importance of the research problems to advancing 
CISE research frontiers, and the expertise of the research team relative 
to the focused research thrust. The description should identify the 
project team and detail each member’s contributions to the project as 
well as specific expertise relative to the proposed focused research 
agenda. Note that Big Data and computational science across a variety of 
disciplines need more specific research focus descriptions including the 
specific data that are involved and available and the specific research 
goals that advance computational science, respectively, as opposed to 
simply employing Big Data and computational science techniques in a broad 
range of applications.

\subsection{Sample research project}
A short (2-3 sentences) summary of one potential research project 
should be included.

\subsection{Nature of the community involvement}
(required for CI-P, CI-New, CI-EN, or CI-SUSTAIN projects). This 
section demonstrates the community involvement in the creation, 
enhancement, evaluation, and use of the resource. Describe the research 
community involved in the project. CI-New projects should show the 
community involvement and demand for the project as well as indicate 
the community commitment to establishing and using the infrastructure. 
CI-EN and CI-SUSTAIN projects should describe community usage of the 
infrastructure as well as community involvement in enhancement and 
sustainment efforts.

\subsection{Relevance to CISE}
This section should include:
\begin{itemize}
  \item A list of specific CISE researchers who are involved in the leadership of the project and in the development of the infrastructure (including any enhancement or sustainment plan, as appropriate);
  \item A list of the CISE researchers communities that will benefit from the infrastructure; and
  \item A list of any prior CISE funding the infrastructure has received (enhancement projects should also include the approximate date when the infrastructure involved was established).
\end{itemize}

