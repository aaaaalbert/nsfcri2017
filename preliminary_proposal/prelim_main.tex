%!TEX root = preliminary_proposal.tex

\section{Project Description}

\explain{Two pages overall}

\begin{description}
  \item[Project Title:] \sysname -- Testbed Construction Kit With No Pronounceable Acronym
  \item[Project Type:] CI-New \albert{or CI-EN}
  \item[Collaborative Partners:] \todo{All collaborative partners should be identified, including PIs and Co-PIs of the collaborative partners with their institutional and departmental affiliations}
  \item[CISE core division:] CNS (Computer and Network Systems)
  \item[Keywords:] \todo{Three keyword descriptors for the research focus that the requested infrastructure will enable.}
\end{description}


\subsection{Overview}
\explain{A concise description of the infrastructure to be developed, enhanced, or
sustained. This includes a description of major equipment needs for the 
project as well as other significant costs. Projects that involve 
enhancements to, or sustainment of, existing infrastructures should 
include information about the existing resources.}

\sysname is a construction kit for testbeds that contains building 
blocks for bootstrapping, operating, and scaling varied types of testbeds.
It comprises (1) a set of reusable, generic implementations of testbed
infrastructure functions (such as a clearinghouse for mediating resource
access) that is open for use to the community; (2) the actual software
that implements these functions which the public
%(including the wider academic community as well as commercial providers)
is free to disseminate
and replicate on other, non-\sysname infrastructure; and (3) the
architectural guidelines that underlie the implemented software,
provided so as to attract contributions of other implementations and
features.

To assume this role, \sysname requires ....

\albert{1. Mostly software to be written/adapted\\
2a. Hardware in terms of....? Servers? (argue that we are exploring deployment
options, distributed vs centralized, etc.)\\
2b. Moar bettar v6 multihomed uplink}


\subsection{CISE Research Focus}
\explain{This section describes the research focus that is enabled by the 
infrastructure, the importance of the research problems to advancing 
CISE research frontiers, and the expertise of the research team relative 
to the focused research thrust. The description should identify the 
project team and detail each member’s contributions to the project as 
well as specific expertise relative to the proposed focused research 
agenda.}

Research focus: Testbeds have become the go-to architectural approach
to exploratory research in CS. We help build better, more capable,
more interoperable testbeds more efficiently.

Expertise: We've done this before, and are still doing it.
E.g., Justin has a history of involvement in network testbeds
(Planet-Lab/Stork, Seattle), and has many current ``become the default
library'' projects going on (TUF, Uptane, in-toto).

Albert has used lots of testbeds, and is helping to run/construct some.

\subsection{Sample research project}
% A short (2-3 sentences) summary of one potential research project 
% should be included.
A research team wants to construct a remote-accessible sensor network
based on Internet-of-Things devices, requiring device discovery, access
mediation, and data backhaul.
Without \sysname, the team is forced to repeat effort done in previous
endeavors (but in a non-reusable fashion): all of the required
components would be constructed
from scratch, and many useful and interesting pieces of functionality
would likely be left out (due to lack of oversight, funding, expertise)
or implemented in an incomplete, insecure fashion.
Instead, building on top of \sysname allows researchers to focus on their
core contribution, and reuse the \sysname infrastructure (either as-is or
replicated on their own resources) for augmented and perhaps
previously-unanticipated functionality.

\subsection{Nature of the community involvement}
\explain{(required for CI-P, CI-New, CI-EN, or CI-SUSTAIN projects). This 
section demonstrates the community involvement in the creation, 
enhancement, evaluation, and use of the resource. Describe the research 
community involved in the project. CI-New projects should show the 
community involvement and demand for the project as well as indicate 
the community commitment to establishing and using the infrastructure. 
CI-EN and CI-SUSTAIN projects should describe community usage of the 
infrastructure as well as community involvement in enhancement and 
sustainment efforts.}

Creation: Collect design and operational experience from other testbeds and operators.

Enhancement: Comes through use and iteration on the design and
implementation, including ours.

Evaluation and use: Invite community to replace their components with
ours, or also start using our components in parallel. Particularly for
new testbed projects, show that we can decrease the time-to-experiment.

Demand: Implicit -- see how many testbeds and similar projects have been funded
over the last years, and how much overlap there was in what they
implemented. Explicit -- people we know seek for and would use this.


\subsection{Relevance to CISE}
\explain{
This section should include:
\begin{itemize}
  \item A list of specific CISE researchers who are involved in the leadership of the project and in the development of the infrastructure (including any enhancement or sustainment plan, as appropriate);
  \item A list of the CISE researchers communities that will benefit from the infrastructure; and
  \item A list of any prior CISE funding the infrastructure has received (enhancement projects should also include the approximate date when the infrastructure involved was established).
\end{itemize}
}

