%!TEX root = preliminary_proposal.tex

\section{Project Description}

\explain{Two pages overall}

\begin{description}
  \item[Project Title:] \sysname -- Mending the fragmented \acrshort{IoT} landscape
  \item[Project Type:] CI-New \albert{or CI-EN}
  \item[Collaborative Partners:] \todo{All collaborative partners should be identified, including PIs and Co-PIs of the collaborative partners with their institutional and departmental affiliations}
  \item[CISE core division:] CNS (Computer and Network Systems)
  \item[Keywords:] \todo{Three keyword descriptors for the research focus that the requested infrastructure will enable.}
\end{description}


\subsection{Overview}
\explain{A concise description of the infrastructure to be developed, enhanced, or
sustained. This includes a description of major equipment needs for the 
project as well as other significant costs. Projects that involve 
enhancements to, or sustainment of, existing infrastructures should 
include information about the existing resources.}

\sysname is an open platform for research on the \acrfull{IoT}
that helps to construct unified methods for sensor access, data analysis,
and remote access and control. \sysname exposes the full \gls{IoT}
stack for experimentation, and thus enables research on new paradigms
for distributed sensing and sensor data fusion, data analysis, and
participatory sensing that today's tightly vertically-integrated
commercial systems do not offer.

In order to assume this function, \sysname comprises
(1) a set of reusable and amendable generic hardware implementations of
\gls{IoT} nodes (or ``Things'') that are openly accessible for
use by the community via the Internet, and that interested parties
can deploy, either to participate in \sysname or to reap its benefits
and start their own deployment;
(2) software that implements the required interfaces to
enable said open use and that the public is free to disseminate
and replicate on other, non-\sysname infrastructure; and
(3) the architectural guidelines that underlie the implementations,
provided so as to attract contributions of both other hardware
designs for Things, and also new software features and services
that are difficult to implement within the restrictions of current
commercial offerings.\\

The equipment needs for \sysname are dominated by hardware required
for experimentation. We plan to deploy \gls{IoT}-enabled devices
at the PIs' institution, and to solicit volunteers to bootstrap an
initial set of networked Things for open use by the scientific community.
The initial deployment can resort to off-the-shelf components
for prototyping, while future development strategies based on
customized hardware are explored in parallel.
To provide a sense of scale, a single prototype node can be built
from a RaspberryPI single-board computer with additional sensor
\glspl{HAT} and an outdoor case for around 100\$.
For simpler or more self-contained tasks where general-purpose
programmability and debugging is less of an issue, an ESP8266
WiFi-networked microcontroller with a custom sensor attachment
might be had for below 5\$ already.

A larger proportion of costs will accrue to building the software
that enables the proposed open, participatory access by the community.
This comprises the sensing and networking code to be run on Things,
the infrastructure for backhauling data, support for concurrent
sensor access by researchers, privacy protection schemes for
volunteers that host Things, and development and deployment tools
that ease the adoption of \sysname throughout the community.


\albert{HW we've tested includes RasPis, routers, smartphones.}
\albert{SW we've done includes sensing and privacy.}
The \acrfull{IoT} paradigm proposes networked embedded devices
for large-scale, distributed sensing, control, and computational tasks,
with the Internet serving as part of the networking substrate for
the devices (``Things'') to exchange information.
In its current realization today, the \acrlong{IoT} comprises
devices in cars, homes and other buildings, public transport,
industrial installations, medical contexts, and farming.
This enables remote control and supervision of systems, provides
means of data collection for evaluation, aids
task automation, and advances convergence of cyber-physical systems.

However, the larger-scale usefulness of current \acrlong{IoT}
systems is held back by design choices of system vendors to
integrate all of the functionality (from sensing and recording
to data exchange and analysis) into monolithic silos of services.
This design not only grants the vendor exclusive access to the
sensor data, but also allows them to exert control over the ways
that devices are used and operated. For example, a firmware update
may disable device functionality (\albert{ref}), or services could
require paid-for subscriptions for the device to work
(\albert{ref}).
If the device owner does not comply, their ability of using
the device might be hampered. Furthermore, the owner has little
choice over which type of device they interlink with what service
provider.

Besides causing inefficiency and loss of control for the end user,
these issues also present substantial hurdles for the progress
of scientific research on distributed sensor devices.
The inability of accessing sensor data precludes using the
data for novel applications; fragmentation and per-vendor silos
prohibit studying cooperative use cases.

Therefore, this proposal suggests to re-think the design choices
that made the current acrlong{IoT} landscape a heap of segregated
monoliths, and instead make it an an open system with proper interfaces and people in control to enable cooperative sensing, crowdery, etc.


\subsection{CISE Research Focus}
\explain{This section describes the research focus that is enabled by the 
infrastructure, the importance of the research problems to advancing 
CISE research frontiers, and the expertise of the research team relative 
to the focused research thrust. The description should identify the 
project team and detail each member’s contributions to the project as 
well as specific expertise relative to the proposed focused research 
agenda.}

Focus: Distributed sensing with an emphasis on openness and participation.

Relevance: Fast-growing field; inaccessibility of paid-for systems.

Expertise: Justin has a history of involvement in network testbeds
(Planet-Lab/Stork, Seattle), and has many current ``become the default
library'' projects going on (TUF, Uptane, in-toto).

Albert has used lots of testbeds, and is helping to run/construct some.
Also EE.


\subsection{Sample research project}
\explain{A short (2-3 sentences) summary of one potential research project 
should be included.}

\subsection{Nature of the community involvement}
\explain{(required for CI-P, CI-New, CI-EN, or CI-SUSTAIN projects). This 
section demonstrates the community involvement in the creation, 
enhancement, evaluation, and use of the resource. Describe the research 
community involved in the project. CI-New projects should show the 
community involvement and demand for the project as well as indicate 
the community commitment to establishing and using the infrastructure. 
CI-EN and CI-SUSTAIN projects should describe community usage of the 
infrastructure as well as community involvement in enhancement and 
sustainment efforts.}


\subsection{Relevance to CISE}
\explain{
This section should include:
\begin{itemize}
  \item A list of specific CISE researchers who are involved in the leadership of the project and in the development of the infrastructure (including any enhancement or sustainment plan, as appropriate);
  \item A list of the CISE researchers communities that will benefit from the infrastructure; and
  \item A list of any prior CISE funding the infrastructure has received (enhancement projects should also include the approximate date when the infrastructure involved was established).
\end{itemize}
}

